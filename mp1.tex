% initialize latex document for CS521 class assignment
\documentclass{article}
\usepackage{amsmath}
\usepackage{amsfonts}
\usepackage{graphicx}
\usepackage{hyperref}
\usepackage{listings}
\usepackage{xcolor}
\usepackage{fancyvrb}   % for verbatim code blocks
\usepackage{geometry}
\geometry{margin=1in}
\usepackage{caption}
\captionsetup{font=small, labelfont=bf}
\usepackage{enumitem}
\setlist[itemize]{left=0pt, labelsep=0.5em, itemsep=0.5em}
\setlist[enumerate]{left=0pt, labelsep=0.5em, itemsep=0.5em}
\usepackage{tcolorbox}
\tcbuselibrary{listingsutf8}
\newtcblisting{pythoncode}{
  listing only,
  colback=white,
  colframe=black,
  listing options={language=Python, basicstyle=\ttfamily\footnotesize, breaklines=true, tabsize=4},
  left=0pt,
  right=0pt,
  top=0pt,
  bottom=0pt,
  boxsep=5pt,
  arc=0pt,
  outer arc=0pt,
  boxrule=0.5pt,
  width=\textwidth,
  before skip=10pt,
  after skip=10pt
}
\title{CS 521: Trustworthy AI Systems \\ MP1}
\author{Adharsh Kamath Raghupathi (netID: ak128)}
\date{}
\begin{document}
\maketitle
\section*{Problem 1:}
\subsection*{Part (1)}
Given the original input x, the target class t, and the perturbation magnitude eps, we can compute FGSM adversary as follows:
\begin{verbatim}
adv_x = x - eps * x.grad.sign()
print("Adversarial Example: ", adv_x)
\end{verbatim}
With this snippet (in the file \href{URL}{text}), we can see the following output:
\begin{verbatim}
Original Input:  tensor([[0.6584, 0.8175, 0.1262, 0.0541, 0.2268, 0.8405, 0.5393, 0.9798, 0.8597,
         0.5977]], requires_grad=True)
Original Class:  2
Adversarial Example:  tensor([[ 0.1584,  1.3175,  0.6262, -0.4459,  0.7268,  0.3405,  0.0393,  0.4798,
          1.3597,  0.0977]], grad_fn=<SubBackward0>)
Adversarial Class:  0
tensor(0.5000)
\end{verbatim}
\subsection*{Part (2)}

\section*{Problem 2:}

\section*{Problem 3:}



\end{document}